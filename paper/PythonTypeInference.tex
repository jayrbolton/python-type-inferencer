\documentclass{article}
\usepackage{amsmath}

\title{Python Type Inference: If It Quacks Like a Duck...}
\author{Jay R Bolton}

\addtolength{\oddsidemargin}{-.875in}
\addtolength{\evensidemargin}{-.875in}
\addtolength{\textwidth}{1.75in}
\addtolength{\topmargin}{-.875in}
\addtolength{\textheight}{1.75in}

\begin{document}
\maketitle

\section*{Introduction}

Python is a wildly dynamic language whose program behavior depends almost
entirely on the execution path. Despite being very multi-paradigm, it is also
not a very complicated language, with an abstract grammar of only around 100
lines. The enforcement of type correctness in python is performed at run-time
using Duck Typing, where we are only concerned with the comparative structure
of two objects. We'll explore this further in sections below.

All languages can be statically typed, no matter how dynamic their current
implementation may be. Our goal here is to perform static type checking (albeit
in a limited form) to a very dynamic language with very minimal type
enforcement. Our type checking system will be optional and will not affect your
ability to run the program. The advantages of having such a system are:

\begin{itemize}
\item Retain `agile development' without limiting possible correct programs.
\item Still be able to analyze the type correctness of your programs, catching errors that you may not find in tests.
\item Provide documentation and meaningful type annotations that can be helpful
in understanding your program in a number of different ways, discussed in the
last section.
\end{itemize}

\section*{The Type System}

Here will be an overview of how python's type system works, with some formalized notation for it.

\section*{Type Inference Using Only Attributes}

First formally define the type system...

\begin{align*}
type &= Obj\ \{ attr : type,\ attr:type,\ \cdots \} \\
     &\ |\ Builtin \\
attr &= name \\
Builtin &= Tuple (type,\ type,\ \cdots) \\
        &\ |\ Int\ |\ Str\ |\ \cdots
\end{align*}

Then define all the inference rules...

\begin{align*}
&Lookup\\
\frac{M\ \vdash\ e:\tau} {M\ \vdash\ e\ :\ \tau} \\
&Assignment\\
\frac{M\ \vdash\ e:\tau} {M\ \vdash\ x = e,\ x\ :\ \tau} \\
&Function\ Definition\\
\frac{M,\text{[params]}:\tau\vdash\ [body]:\{`*r':\sigma\}} {M\ \vdash\ \text{def f([params]) [body]}\ :\ \{`*r':\sigma,\ `*p': \tau\}}
\end{align*}

\section*{Examples and Cases}

Present an example program and walk through all the type inference rules. 

Then discuss the edge cases and limitations of inferencing python, such as dynamic redefinition of attributes.

\section*{Improvement and Expansion}

Talk about how the system could be expanded and all its potential uses.

\end{document}
